%\thispagestyle{empty} 
%\pagestyle{empty}
\fancyhf{}
\lhead{Research and seminar reports}
\cfoot{\thepage}

\chapter*{Research and seminar reports}

\begin{itemize}
    \item[S 73/1] Seminar über Thermodynamik und Kontinuumsmechanik, Hannover 1973.
    \item[F 75/1] ``Die Spannungsberechnung im Rahmen der Finite-Element-Methode”, R. Ahmad, Dissertation, April 1975.
    \item[F 76/1] ``Zur Theorie und Anwendung der Stoffgleichungen elastisch-plastisch- viskoser Werkstoffe”, H. Mentlein, Dissertation, April 1976.
    \item[F 77/1] Seminar über lineare und geometrisch nichtlineare Schalentheorie einschließlich Stabilit\"atstheorie, Hannover 1978.
    \item[F 77/2] ``Beitrag zur Berechnung von Gründungsplatten mit Hilfe der Finite- Element-Methode”, H. Meyer, Dissertation, Juli 1977.
    \item[F 77/3] ``Zur Berechnung der Eigenfrequenzen und Eigenschwingungsformen räumlich vorgekrümmter und vorverwundener Stäbe” J. M\"ohlenkamp, Dissertation, Dezember 1977.
    \item[F 77/4] ``Zur Theorie und Berechnung geometrisch und physikalisch nichtlinearer Kontinua mit Anwendung der Methode der finiten Elemente”, J. Paulun, Dissertation, Dezember 1977.
    \item[F 78/1] 2. Seminar über Thermodynamik und Kontinuumsmechanik, Hannover 1978.
    \item[F 79/1] ``Theoretische und numerische Behandlung geometrisch nichtlinearer viskoplastischer Kontinua”, K.-D. Klee, Dissertation, Februar 1979.
    \item[F 79/2] ``Zur Konstruierbarkeit von Variationsfunktionalen für nichtlineare Probleme der Kontinuumsmechanik”, J. Siefer, Dissertation, Oktober 1979.
    \item[F 80/1] ``Theoretische und numerische Behandlung gerader Stäbe mit endlichen Drehungen”, M. Kessel, Dissertation, Februar 1980.
    \item[F 81/1] ``Zur Berechnung von Kontakt- und Stoßproblemen elastischer Körper mit Hilfe der Finite-Element-Methode”, P. Wriggers, Dissertation, Januar 1981.95
    \item[F 81/2] ``Stoffgleichungen für Steinsalze unter mechanischer und thermischer Beanspruchung”, J. Olschewski, E. Stein, W. Wagner, D. Wetjen, geänderte Fassung eines Zwischenberichtes zum BMFT-Forschungsvorhaben KWA 1608/5.
    \item[F 82/1] ``Konvergenz und Fehlerabschätzung bei der Methode der Finiten Elemente”, R. Rohrbach, E. Stein, Abschlußbericht eines VW- Forschungsvorhabens, Februar 1982.
    \item[F 82/2] ``Alternative Spannungsberechnung in Finite-Element- Verschiebungsmodellen”, C. Klöhn, Dissertation, November 1982
    \item[F 83/1] Seminar über nichtlineare Stabtheorie, Hannover 1983.
    \item[F 83/2] ``Beiträge zur nichtlinearen Theorie und inkrementellen Finite- Element-Berechnung dünner elastischer Schalen”, A. Berg, Dissertation, Juli 1983.
    \item[F 83/3] ``Elastoplastische Plattenbiegung bei kleinen Verzerrungen und großen Drehungen”, J. Paulun, Habilitation, September 1983.
    \item[F 83/4] ``Geometrisch nichtlineare FE-Berechnung von Faltwerken mit plastisch / viskoplastischem Deformationsverhalten”, M. Krog, Dissertation, Dezember 1983.
    \item[F 85/1] Verleihung der Ehrendoktorwürde des Fachbereichs Bauingenieur- und Vermessungswesen der Universität Hannover an die Herren Prof. Dr. Drs. h.c. J.H. Argyris, Dr.-Ing. H. Wittmeyer.
    \item[F 85/2] ``Eine geometrisch nichtlineare Theorie schubelastischer Schalen mit Anwendung auf Finite-Element-Berechnungen von Durchschlag- und Kontaktproblemen”, W. Wagner, Dissertation, März 1985.
    \item[F 85/3] ``Geometrisch/physikalisch nichtlineare Probleme - Struktur und Algorithmen - ”, GAMM-Seminar im Februar 1985 in Hannover.
    \item[F 87/1] ``Finite-Elemente-Berechnungen ebener Stabtragwerke mit Fließgelenken und großen Verschiebungen”, R. Kahn, Dissertation, Oktober 1987.
    \item[F 88/1] ``Theorie und Numerik schubelastischer Schalen mit endlichen Drehungen unter Verwendung der Biot-Spannungen”, F. Gruttmann, Dissertation, Juni 1988.
    \item[F 88/2] ``Optimale Formgebung von Stabtragwerken mit Nichtlinearitäten in der Zielfunktion und in den Restriktionen unter Verwendung der Finite-Element-Methode”, V.Berkhahn, Dissertation, Oktober 1988.
    \item[F 88/3] ``Beiträge zur Theorie und Numerik großer plastischer und kleiner elastischer Deformationen mit Schädigungseinfluß”, R. Lammering, Dissertation, November 1988.
    \item[F 88/4] ``Konsistente Linearisierungen in der Kontinuumsmechanik und ihrer Anwendung auf die Finite-Elemente-Methode”, P. Wriggers, Habilitation, November 1988.96
    \item[F 88/5] ``Mathematische Formulierung und numerische Methoden für Kontaktprobleme auf der Grundlage von Extremalprinzipien”, D. Bischoff, Habilitation, Dezember 1988.
    \item[F 88/6] ``Zur numerischen Behandlung thermomechanischer Prozesse”, C. Miehe, Dissertation, Dezember 1988.
    \item[F 89/1] ``Zur Stabilität und Konvergenz gemischter finiter Elemente in der linearen Elastizitätstheorie”, R. Rolfes, Dissertation, Juni 1989.
    \item[F 89/2] ``Traglastberechnungen von Faltwerken mit elastoplastischen Deformationen”, K.-H. Lambertz, Dissertation, November 1989.
    \item[F 89/3] ``Transientes Kriechen und Kriechbruch im Steinsalz”, U. Heemann, Dissertation, November 1989.
    \item[F 89/4] ``Materialgesetze zum Verhalten von Betonkonstruktionen bei harten Stößen”, E. Stein, P. Wriggers, T. Vu Van \& T. Wedemeier, Dezember 1989.
    \item[F 89/5] ``Lineare Konstruktion und Anwendungen von Begleitmatrizen”, C. Carstensen, Dissertation, Dezember 1989.
    \item[F 90/1] ``Zur Berechnung prismatischer Stahlbetonbalken mit verschiedenen Querschnittformen für allgemeine Beanspruchungen”, H. N. Lucero-Cimas, Dissertation, April 1990.
    \item[F 90/2] ``Zur Behandlung von Stoß- Kontaktproblemen mit Reibung unter Verwendung der Finite-Element-Methode”, T. Vu Van, Dissertation, Juni 1990.
    \item[F 90/3] ``Netzadaption und Mehrgitterverfahren für die numerische Behandlung von Faltwerken”, L. Plank, Dissertation, September 1990.
    \item[F 90/4] ``Beiträge zur Theorie und Numerik finiter inelastischer Deformationen”, N. Müller-Hoeppe, Dissertation, Oktober 1990.
    \item[F 90/5] ``Beiträge zur Theorie und Numerik von Materialien mit innerer Reibung am Beispiel des Werkstoffes Beton”, T. Wedemeier, Dissertation, Oktober 1990.
    \item[F 91/1] ``Zur Behandlung von Stabilitätsproblemen der Elastostatik mit der Methode der Finiten Elemente”, W. Wagner, Habilitation, April 1991.
    \item[F 91/2] ``Mehrgitterverfahren und Netzadaption für lineare und nichtlineare statische Finite-Elemente-Berechnungen von Flächentragwerken”, W. Rust, Dissertation, Oktober 1991.
    \item[F 91/3] ``Finite Elemente Formulierung im Trefftzschen Sinne für dreidimensionale anisotrop-elastische Faserverbundstrukturen”, K. Peters, Dissertation, Dezember 1991.
    \item[F 92/1] ``Einspielen und dessen numerische Behandlung von Flächentragwerken aus ideal plastischem bzw. kinematisch verfestigendem Material”, G. Zhang, Dissertation, Februar 1992.97
    \item[F 92/2] ``Strukturoptimierung stabilitätsgefährdeter Systeme mittels analytischer Gradientenermittlung”, A. Becker, Dissertation, April 1992.
    \item[F 92/3] ``Duale Methoden für nichtlineare Optimierungsprobleme in der Strukturmechanik”, R. Mahnken, Dissertation, April 1992.
    \item[F 93/1] ``Kanonische Modelle multiplikativer Elasto-Plastizität. Thermodynamische Formulierung und numerische Implementation”, C. Miehe, Habilitation, Dezember 1993.
    \item[F 93/2] ``Theorie und Numerik zur Berechnung und Optimierung von Strukturen aus isotropen, hyperelastischen Materialien”, F.-J. Barthold, Dissertation, Dezember 1993.
    \item[F 94/1] ``Adaptive Verfeinerung von Finite-Element-Netzen für Stabilitätsprobleme von Fläschentragwerken”, E. Stein, B. Seifert, W. Rust, Forschungsbericht, Oktober 1994.
    \item[F 95/1] ``Adaptive Verfahren für die Formoptimierung von Flächentragwerken unter Berücksichtigung der CAD-FEM-Kopplung”, A. Falk, Dissertation, Juni 1995.
    \item[F 96/1] ``Theorie und Numerik dünnwandiger Faserverbundstrukturen”, F. Gruttmann, Habilitation, Januar 1996.
    \item[F 96/2] ``Zur Theorie und Numerik finiter elastoplastischer Deformationen von Schalenstrukturen”, B. Seifert, Dissertation, März 1996.
    \item[F 96/3] ``Theoretische und algorithmische Konzepte zur phänomenologischen Beschreibung anisotropen Materialverhaltens”, J. Schröder, Dissertation, März 1996.
    \item[F 96/4] ``Statische und dynamische Berechnungen von Schalen endlicher elastischer Deformationen mit gemischten finiten Elementen”, P. Betsch, Dissertation, März 1996.
    \item[F 96/5] ``Kopplung von Finiten Elementen und Randelementen für ebene Elastoplastizität mit Impelementierung auf Parallelrechnern”, M. Kreienmeyer, Dissertation, März 1996.
    \item[F 96/6] ``Theorie und Numerik dimensions- und modeladaptiver Finite- Elemente-Methoden von Fläschentragwerken”, S. Ohnimus, Dissertation, Juni 1996.
    \item[F 96/7] ``Adaptive Finite Elemente Methoden für MIMD-Parallelrechner zur Behandlung von Strukturproblemen mit Anwendung auf Stabilitätsprobleme”, O. Klaas, Dissertation, Juli 1996.
    \item[F 96/8] ``Institutsbericht 1971-1996 aus Anlaß des 25-jährigen Dienstjubiläums von Prof. Dr.-Ing. Dr.-Ing. E.h. Dr. h.c. mult. Erwin Stein”, Dezember 1996.
    \item[F 97/1] ``Modellierung und Numerik duktiler kristalliner Werkstoffe”, P. Steinmann, Habilitation, August 1997.
    \item[F 97/2] ``Formoptimierung in der Strukturdynamik”, L. Meyer, Dissertation, September 1997.98
    \item[F 97/3] ``Modellbildung und Numerik für Versagensprozesse in Gründungen von Caisonwellenbrechern”, M. Lengnick, Dissertation, November 1997.
    \item[F 98/1] ``Adaptive gemischte finite Elemente in der nichtlinearen Elastostatik und deren Kopplung mit Randelementen”, U. Brink, Dissertation, Februar 1998.
    \item[F 98/2] ``Theoretische und numerische Aspekte zur Parameteridentifikation und Modellierung bei metallischen Werkstoffen”, R. Mahnken, Habilitation, Juli 1998.
    \item[F 98/3] ``Lokalisierung und Stabilität der Deformation wassergesättigter bindiger und granularer Böden”, J. M. Panesso, Dissertation, August 1998.
    \item[F 98/4] ``Theoretische und numerische Methoden in der angewandten Mechanik mit Praxisbeispielen”, R. Mahnken (Hrsg.), Festschrift anlässlich der Emeritierung von Prof. Dr.-Ing. Dr.-Ing. E.h. h.c. mult. Erwin Stein, November 1998.
    \item[F 99/1] ``Eine h-adaptive Finite-Element-Methode für elasto-plastische Schalenproblem in unilateralem Kontakt”, C.-S. Han, Dissertation, Juli 1999.
    \item[F 00/1] ``Ein diskontinuierliches Finite-Element-Modell für Lokalisierungsversagen in metallischen und granularen Materialien”, C. Leppin, Dissertation, März 2000.
    \item[F 00/2] ``Untersuchungen von Strömungen in zeitlich veränderlichen Gebieten mit der Methode der Finiten Elementen”, H. Braess, Dissertation, März 2000.
    \item[F 00/3] ``Theoretische und algorithmische Beiträge zur Berechnung von Faserverbundschalen”, J. Tessmer, Dissertation, März 2000.
    \item[F 00/4] ``Theorie und Finite-Element-Methode für die Schädigungsbeschreibung in Beton und Stahlbeton”, D. Tikhomirov, Dissertation, August 2000.
    \item[F 01/1] ``A C1 - continuous formulation for finite deformation contact”, L. Krstulovic-Opara, Dissertation, Januar 2001.
    \item[F 01/2] ``Strain Localisation Analysis for Fully and Partially Saturated Geomaterials”, H. Zhang, Dissertation, Januar 2001.
    \item[F 01/3] ``Meso-makromechanische Modellierung von Faserverbundwerkstoffen mit Schädigung”, C. Döbert, Dissertation, April 2001.
    \item[F 01/4] ``Thermomechanische Modellierung gummiartiger Polymerstrukturen”, S. Reese, Habilitation, April 2001.
    \item[F 01/5] ``Thermomechanisches Verhalten von Gummimaterialien während der Vulkanisation - Theorie und Numerik -”, M. Andre, Dissertation, April 2001.
    \item[F 01/6] ``Adaptive FEM für elastoplastische Deformationen - Algorithmen und Visualisierung”, M. Schmidt, Dissertation, Juni 2001.
    \item[F 01/7] ``Verteilte Algorithmen für h-, p- und d-adaptive Berechnungen in der nichtlinearen Strukturmechanik”, R. Niekamp, Dissertation, Juni 2001.99
    \item[F 01/8] ``Theorie und Numerik zur Berechnung und Optimierung von Strukturen mit elastoplastischen Deformationen”, K. Wiechmann, Dissertation, Juli 2001.
    \item[F 01/9] ``Direct Computation of Instability Points with Inequality using the Finite Element Method”, H. Tschöpe, Dissertation, September 2001.
    \item[F 1/10] ``Theorie und Numerik residualer Fehlerschätzer für die Finite- Elemente-Methode unter Verwendung äquilibrierter Randspannungen”, S. Ohnimus, Habilitation, September 2001.
    \item[F 02/1] ``Adaptive Algorithmen für thermo-mechanisch gekoppelte Kontaktprobleme”, A. Rieger, Dissertation, August 2002.
    \item[F 02/2] ``Consistent coupling of shell- and beam-models for thermo-elastic problems”, K.Chavan, Dissertation, September 2002.
    \item[F 03/1] ``Error-controlled adaptive finite element methods in large strain hyperelasticity and fracture mechanics”, M. Rüter, Dissertation, Mai 2003.
    \item[F 03/2] ``Formulierung und Simulation der Kontaktvorgänge in der Baugrund- Tragwerks- Interaktion”, A. Haraldsson, Dissertation, Juni 2003.
    \item[F 03/3] ``Concepts for Nonlinear Orthotropic Material Modeling with Applications to Membrane Structures”, T. Raible, Dissertation, Juni 2003.
    \item[F 04/1] ``On Single- and Multi-Material arbitrary Lagrangian-Eulerian Approaches with Application to Micromechanical Problems at Finite Deformations”, D. Freßmann, Dissertation, Oktober 2004.
    \item[F 04/2] ``Computational Homogenization of Microheterogeneous Materials at Finite Strains Including Damage”, S. Löhnert, Dissertation, Oktober 2004.
    \item[F 05/1] ``Numerical Micro-Meso Modeling of Mechanosensation driven Osteonal Remodeling in Cortical Bone”, C. Lenz, Dissertation, Juli 2005.
    \item[F 05/2] ``Mortar Type Methods Applied to Nonlinear Contact Mechanics”, K.A. Fischer, Dissertation, Juli 2005.
    \item[F 05/3] ``Models, Algorithms and Software Concepts for Contact and Fragmentation in Computational Solid Mechanics”, C. Hahn, Dissertation, November 2005.
    \item[F 06/1] ``Computational Homogenization of Concrete”, S. Moftah, Dissertation, Januar 2006.
    \item[F 06/2] ``Reduction Methods in Finite Element Analysis of Nonlinear Structural Dynamics”, H. Spiess, Dissertation, Februar 2006.
    \item[F 06/3] ``Theoretische und algorithmische Konzepte zur Beschreibung des beanspruchungsadaptiven Knochenwachstums”, B. Ebbecke, Dissertation, März 2006.100
    \item[F 06/4] ``Experimentelle Untersuchungen an elastomeren Werkstoffen”, M. Dämgen, Dissertation, Dezember 2006.
    \item[F 07/1] ``Numerische Konzepte zur Behandlung inelastischer Effekte beim reibungsbehafteten Rollkontakt”, M. Ziefle, Dissertation, Februar 2007.
    \item[F 07/2] ``Begleitbuch zur Leibniz-Ausstellung”, Hrsg: E. Stein, P. Wriggers, 2007.
    \item[F 07/3] ``Modellierung und Simulation der hochfrequenten Dynamik rollender Reifen”, M. Brinkmeier, Dissertation, Juni 2007.
    \item[F 07/4] ``Computational Homogenization of micro-structural Damage due to Frost in Hardened Cement Paste”, M. Hain, Dissertation, Juli 2007.
    \item[F 07/5] ``Elektromechanisch gekoppelte Kontaktmodellierung auf Mikroebene”, T. Helmich, Dissertation, August 2007.
    \item[F 07/6] ``Dreidimensionales Diskretes Elemente Modell für Superellipsoide”, C. Lillie, Dissertation, Oktober 2007.
    \item[F 07/7] ``Adaptive Methods for Continuous and Discontinuous Damage Modeling in Fracturing Solids”, S.H. Reese, Dissertation, Oktober 2007.
    \item[F 08/1] ``Student Projects of Micromechanics”, Hrsg: U. Nackenhorst, August 2008.
    \item[F 09/1] ``Theory and Computation of Mono- and Poly- crystalline Cyclic Martensitic Phase Transformations”, G. Sagar, Dissertation, August 2009.
    \item[F 09/2] ``Student projects of Micromechanics”, D. Balzani and U. Nackenhorst, Course Volume, Oktober 2009.
    \item[F 09/3] ``Multiscale Coupling based on the Quasicontinuum Framework, with Application to Contact Problems”, W. Shan, Dissertation, November 2009.
    \item[F 10/1] ``A Multiscale Computational Approach for Microcrack Evolution in Cortical Bone and Related Mechanical Stimulation of Bone Cells”, D. Kardas, Dissertation, September 2010. 
    \item[F 11/1] ``Ein Integrales Modellierungskonzept zur numerischen Simulation der Osseointegration und Langzeitstabilität von Endoprothesen”, A.Lutz, Dissertation, Oktober 2011.
    \item[F 12/1] ``Ein physikalisch motiviertes Reifen-Fahrbahnmodell für die Gesamtfahrzeugsimulation”, R. Chiarello, Dissertation, Februar 2012.
    \item[F 13/1] ``Thermomechanical Analysis of Tire Rubber Compounds in Rolling Contact”, A.Suwannachit, Dissertation, September 2012.
    \item[F 13/2] ``Towards a Finite Element Model for Fluid Flow in the Human Hip Joint”, K. Fietz, Dissertation, September 2013.101
    \item[F 14/1] ``Micro-Mechanically Based Damage Analysis of Ultra High Performance Fibre Reinforced Concrete Structures with Uncertainties”, A. Hürkamp, Dissertation, Dezember 2013.
    \item[F 14/2] ``Numerical Solution of High-Dimensional Fokker-Planck Equations with Discontinuous Galerkin Methods”, F. Loerke, Dissertation, Dezember 2013.
    \item[F 14/3] ``Numerische Simulation probabilistischer Schädigungsmodelle mit der Stochastischen Finite Elemente Methode”, P. Jablonski, Dissertation, September 2014.
    \item[F 15/1] ``On a Finite Element Approach for the Solution of a Mechanically Stimulated Biochemical Fracture Healing Model”, A. Sapotnick, Dissertation, November 2015.
    \item[F 15/2] ``Simulation of Elastic-Plastic Material Behaviour with Uncertain Material Parameters. A Spectral Stochastic Finite Element Method Approach”, S. Fink, Dissertation, November 2015.
    \item[F 15/3] ``A Fully Micro-mechanically Motivated Material Law for Filled Elastomer”, O.Stegen, Dissertation, Februar 2016.
    \item[F 16/1] ``A modified adaptive harmony search algorithm approach on structural identification and damage detection”, M.Jahjouh, Dissertation, Januar 2016,
    \item[F17/1] ``Computation Simulation of Piezo-electrically Stimulated Bone Adaption Surrounding Activated Teeth Implants”, A.Shirazibeheshtiha, Dissertation, Januar 2017.
    \item[F 17/2] ``A Constitutive Contact Model for Homogenized Tread-Road Interaction in Rolling Resistance Computations”, R.Bayer, Dissertation, Februar 2017.
    \item[F 17/3] ``A Posteriori Error Estimates for Advanced Galerkin Methods”, M.O. Rüter, Habilitation, November 2017.
    \item[F 17/4] ``Probabilistische Finite Element Modellierung des mechanischem Materialverhaltens von Salzgestein”, M. Grehn, Dissertation, Dezember 2017.
    \item[F 18/1] ``Modelling and numerical simulation for the prediction of the fatigue strength of airsprings”, N.K.Jha, Dissertation, März 2018.
    \item[F 18/2] ``A model reduction approach in space and time for fatigue damage simulation”, M.Bhattacharya, Dissertation, Mai 2018.
    \item[F 18/3] ``Numerical investigation on hydrogen embrittlement of metallic pipeline structures”, M.Möhle, Dissertation, Mai 2018.
    \item[F 18/5] ``A stochastic fatigue model for casted aluminium structures", Govindarajan Narayanan, Dissertation, August 2018.
    \item[F 20/1] ``A Micro-mechanically Motivated Approach for Modelling the Oxidative Aging Process of Elastomers", Darcy Beurle, Dissertation, December 2019.
\end{itemize}
