\documentclass{article}
\usepackage{hyperref}
\usepackage[T1]{fontenc}

\setlength{\parindent}{0pt}
\setlength{\parskip}{1ex plus 0.5ex minus 0.2ex}

\begin{document}
\title{Introduction of new commands for the \texttt{w-art.cls}
  classfile}
\maketitle

\section{Commands}


The following commands will have no effect on the visual representation of the
resulting pdf file. They will only be used by scripts to ease and automate the
export of specific data that will then be used for the web presentation of the
article. The country and language codes XX, EN, FR, DE, and GB are only
examples. Please replace them according to your needs.

\subsection{Title}


\verb+\TitleLanguage[XX]+


Indicates the language in which the title of the article appears, using a
two-letter language code from ISO 639:1988. The most common codes are ``EN''
(English; the default language), ``DE'' (German), and ``FR'' (French). 

See \url{http://en.wikipedia.org/wiki/List_of_ISO_639-1_codes}

\paragraph{Note:} The language code applies both to the title and the optional
shorttitle.  The title should be in sentence case, i.e., the first letter of
the first word is (usually) capitalized, while others are all lower case
except for names, etc., as follows:

\begin{verbatim}
\TitleLanguage[EN]
\title[The short title]{This is the long form of the article title}
\end{verbatim}
Please  use always upper case letters for the language code. The command
\verb+\TitleLanguage[XX]+ should be placed \emph{before and outside} the \verb+\title+ command.


\subsection{Abstract}

\verb+\AbstractLanguage[XX]+


Specifies the language in which the abstract was written, using a two-letter
language code from ISO 639:1988. The most common codes are ``EN''
(English; the default language), ``DE'' (German), and ``FR'' (French). 

See \url{http://en.wikipedia.org/wiki/List_of_ISO_639-1_codes}

Usage:
\begin{verbatim}
\AbstractLanguage[EN]
\begin{abstract}
  Text of the abstract.
\end{abstract}
\end{verbatim}
Please  use always upper case letters for the language code. The command
\verb+\AbstractLanguage[XX]+ should be placed \emph{before and outside} the \verb+\abstract+ environment.



\subsection{Author}

There are four new macros that should be used \emph{inside} the
\verb+\author{}+ command:
\begin{itemize}
\item \verb+\firstname{First}+ (unabbreviated forename, given name, or Christian name)

\item \verb+\lastname{Last}+ (surname, family name, or second name)

\item \verb+\namesuffix{Jr}+ is used for family qualifications, such as 
Jr, Sr, III, etc.
\item \verb+\ElectronicMail{xx@yyy.zz}+ is used \emph{inside} the
  \verb+\footnote+ command of the \verb+\author+ command in order to mark the
  e-mail address of the respective author. The command
  \verb+\ElectronicMail{xx@yyy.zz}+ should only be used at this place and not
  in the text section.
\end{itemize}
Usage:
\begin{verbatim}
\author{\firstname{First} \lastname{Last}\inst{1,}%
\footnote{Corresponding author: email \ElectronicMail{x.y@xxx.yyy.zz}}}
\end{verbatim}


\subsection{Address}

\verb+\CountryCode[XX]+

Two-letter ISO 3166 code indicating the country of the author's affiliated
organization or institution.

See \url{http://en.wikipedia.org/wiki/ISO_3166-1}

The command \verb+\CountryCode[GB]+ should be placed \emph{inside} the
\verb+\address{}+ command. Please note that country and language codes may
differ, e.g., GB and EN.

Usage:

\verb+\address[\inst{1}]{\CountryCode[DE]First address}+
\subsection{Relations between Authors and Addresses}

To show the relations between authors and addresses, superscript numbers
created by the \verb+\inst{}+ command are used. The numbers have to be 
entered by the author. A simple example could be:
\begin{verbatim}
\author{\firstname{First} \lastname{Author}\inst{1}}
\address[\inst{1}]{\CountryCode[XX]First address}
\author{\firstname{Second} \lastname{Author}\inst{1}}
\end{verbatim}
Here, both authors are marked with a superscript 1 and the address  will also
be marked by superscript 1. In both cases the superscript will be created by
the \verb+\inst{1}+ command. In case of the \verb+\address+ command, the
\verb+\inst{1}+ command will be placed in the optional part of the
\verb+\address+ command, i.e. \verb+\address[\inst{1}]{...}+. 

One author may have more than one address. For instance
\begin{verbatim}
\author{\firstname{First} \lastname{Author}\inst{1,2}}
\address[\inst{1}]{\CountryCode[XX]First address}
\author{\firstname{Second} \lastname{Author}\inst{1}}
\address[\inst{2}]{\CountryCode[YY]Second address}
\end{verbatim}
Here, the first author has address 1 and 2. The second author has address 1.

The information for the corresponding author and/or the e-mail address of an
author will be placed in a footnote inside the \verb+\author{}+ command. The
footnote follows immediately after the \verb+\inst{}+ marker of the
\verb+\author{}+ command.
\begin{verbatim}
\author{name part\inst{1,}\footnote{\ElectronicMail{x.y@xxx.yyy.zz}}}
\end{verbatim}
If a footnote follows after the \verb+\inst{}+ command, a comma has to be put
inside the \verb+\inst{1,}\footnote{...}+ command. (This will then be rendered
as $^{1,*}$.)

Usually there is  only one \textit{corresponding author} per article. You may, however,
give the e-mail addresses for the other authors in their respective footnotes too.

\subsection{General remarks}

\begin{itemize}

\item Please do not use self-defined macros inside the title
  or the abstract. This restriction is necessary because the title and
  abstract text will be exported and used in a web page for the article. The
  rendering of the \LaTeX-parts is done using a minimal \LaTeX-installation
  where no user defined macros are known.

\item Please define your self-defined macros inside the preamble of your
  \LaTeX-document. The preamble is the place immediately after the
  \verb+\documentclass+ line and before the \verb+\begin{document}+ 
 line. 

\end{itemize}

Happy LaTeXing!

\end{document}
